% !TeX root = ../main.tex

\begin{acknowledgements}

感谢中国科学院计算技术研究所的张曦珊老师、张蕊老师、贾京凯老师和陈云霁老师。张曦珊老师为我提供了毕业设计的选题以及图神经网络相关背景
知识的指导,同时张曦珊老师为我提供了前往中科院计算所完成毕设的机会,让我能够使用到宝贵的实验数据和显卡设备,
张蕊老师帮助我拟定论文大纲,并对论文以及相关汇报工作提供了修改意见。贾京凯老师帮助我
办理了赴京进行毕设的相关手续。陈云霁老师为我提供了中科院计算所的读研机会。

感谢中科院计算所的王鹏程师兄和何得园师兄。王鹏程师兄在相关背景知识的调研,实验数据的处理以及论文的撰写等方方面面
均给我提供了巨大的帮助,最重要的是王鹏程师兄为我提供了初版的数据处理程序,从而极大地减轻了我的工作量,在此特别表示感谢。
何得园师兄对我的论文提出了修改意见,两位师兄均对图神经网络具体结构的设计提供了指导。

感谢中国科学技术大学计算机学院的张昱老师、李诚老师和王超老师。在大四上学期担任张昱老师的编译原理课程助教的过程中,我巩固了编译优化的基础知识,这帮助我
在完成毕设的过程中快速理解深度学习编译技术的许多基本概念。李诚老师和王超老师为我介绍了中科院计算所的智能处理器研究中心,并在保研的过程中为我提供了推荐,
这间接帮助我完成了本毕设。李诚老师作为我本科的班主任,在学习和生活上为我提供了很多帮助。

感谢中国科学技术大学计算机学院的薛佳老师和宫晓梅老师。薛佳老师为我办理了赴京进行毕设的相关手续,
同时在我完成毕设期间远程帮助我完成一些学校交代的任务,解决了我的后顾之忧。我因为签署客座学生协议的问题耽误了
宫晓梅老师很多时间,同时也使她承担了一些不必要的风险,在此表达歉意。

感谢中国科学技术大学计算机学院的龚睿杰同学、张阳同学和陈文章同学。龚睿杰同学曾在我处于危难之际,不顾自身安危,挺身
而出,给予了我极大的鼓励与支持。在我赴京完成毕设期间,张阳同学帮我处理学校的各项事务。龚睿杰同学和张阳同学作为我
四年的室友,在学习和生活上为我提供了极大的帮助。陈文章同学与我交流探讨了一些计算机问题,为我完成毕设提供了帮助。

大恩不言谢,唯有继续努力才能回报以上帮助过我的人。

\end{acknowledgements}
