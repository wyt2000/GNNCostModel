% !TeX root = ../main.tex

\chapter{总结与展望}
本文从深度学习编译优化的相关背景出发,介绍了计算图耗时预测问题及其传统方法,以及本文设计的
一种基于图神经网络的计算图耗时预测模型,并通过实验证明了该模型的有效性。

本文设计的模型首先使用正则表达式从 Y 编译器的输入文件 MLIR 和 Protobuf 中提取计算图
结构信息以及结点和边的相关特征信息。随后将结点和边的特征分为枚举特征和数值特征两类,对于
枚举特征,本文将其编码为相互正交的独热码;对于数值特征,本文使用减去均值除以标准差的方法
将其归一化,从而得到特征向量。

本文使用全连接神经网络将特征向量嵌入到低维向量空间中,同时初步提取其特征。低维嵌入向量以及
图结构信息将作为图神经网络的输入。本文交替使用能够处理边特征的图神经网络和在多个结点之间传
递消息的传统图神经网络聚合结点和边的信息,并归约结点嵌入以得到整个计算图的嵌入向量,从而
用于预测计算图执行时间。

本文进行了多组实验对比传统方法、本文所使用模型的各个子模块和完整模型。这些实验成功证明了本文提出
的模型可以很好地进行计算图耗时预测,同时证明了图神经网络在本文提出的模型中起到了至关重要的作用。

本文的模型仍存在很多不足。首先,该模型虽然在整体的肯德尔秩相关系数上取得了不错的结果,但
每个 IR 的肯德尔秩相关系数仍然存在一定的提升空间,而深度学习编译器实际上需要的是确定每个
IR 应该采取哪种硬件配置才能得到更短的运行时间,不同 IR 之间的运行时间排序其实并不重要。

其次,本文的模型对数据的依赖性较强。本文实验处理的 ResNet-50 数据集是经过打乱后划分为
训练集、验证集和测试集的,故测试集中会存在部分数据与训练集类似。而如果按照顺序划分数据
集,则预测的效果会变差。原因是模型在测试集中会遇到较多从未见过的数据模式,这说明该模型
的泛化性存在一定的问题。

作者会在未来的工作中进一步解决上述问题。